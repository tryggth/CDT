\documentclass{article}

\author{Jonah Miller\\
\textit{jonah.maxwell.miller@gmail.com}}
\title{2+1-dimensional Fixed Boundaries CDT: Users Guide}

\usepackage{fullpage}
\usepackage{verbatim}
\usepackage{listings}
\usepackage{color}
%\usepackage{algorithmic}

\begin{document}
\maketitle

\section{Introduction}

This is the fixed boundaries branch of Rajesh Kommu's CDT code. The
algorithm was originally worked on by David Kemansky, and later
improved upon and updated by Jonah Miller.

The code implements CDT using an updated action that includes the
boundary terms.

The goal of this file is to explain to the educated user how to run a
simulation. I will focus only on the basics of the user interface for
the LISP code. I have added a large number of python and shell scripts
as well, which I will not discuss in detail here. Each script should
be well commented, and there should be a brief description of use in
the file called ``program\_and\_module\_list.txt''. If you have any
questions, please feel free to email me.

\section{Running the Code}

The code is typically run by creating a Common Lisp script and running
it using the command

\lstset{language=Bash,stepnumber=2}
\begin{lstlisting}
nohup nice sbcl --dynamic-space-size 2000 --script <your-script> >> logname.log &
\end{lstlisting}
In order, each command does the following:
\begin{itemize}
\item nohup---stands for ``no hangup.'' Prevents the program from
  stopping if you log off of your computer.
\item nice---Tells the computer to prioritize this program less highly
  than other programs you start. What this means is that programs
  without the ``nice'' flag will take more cputime than programs that
  have the ``nice'' flag. This prevents the computer from freezing up
  if all the cores are running simulations.
\item sbcl---Steele Bank Common Lisp. The Common Lisp interpeter we use.
\item dynamic-space-size 2000--- This tells the lisp interpreter to
  reserve 2 GB of RAM for use with the garbage collector. If the we
  don't reserve enough memory, the program crashes.
\item script [your script name]---This is hopefully self-explanatory.
\item $>>$ logname.log---This sends the
  output of the program from the console to the file logname.log.
\item \&---This ``daemonizes'' the program, so you can still use your console.
\end{itemize}

It is also possible to run lisp in a live interpreted environment
using common lisp's REPL (read-eval-print-loop), which compiles
individual functions as they're defined and runs them in an
interpreted environment. This is very nice for debugging and
prototyping. To do this, just type
\begin{lstlisting}
sbcl
\end{lstlisting}
into the command line.

Alternatively, you can use SLIME, an extension for emacs, as a more
powerful interpreter and indeed IDE. Slime is a very powerful tool and
I suggest you check it out:
\begin{verbatim}
http://common-lisp.net/project/slime/
\end{verbatim}

\section{Dependencies}
One of the first lines of your script should be 
\lstset{language=Lisp,numbers=left,stepnumber=1,frame=shadowbox,rulesepcolor=\color{blue}}
\begin{lstlisting}
(load ``cdt2p1.lisp'')
\end{lstlisting}
This is a top level wrapper which includes all the most important
modules. It is equivalent to:
\begin{lstlisting}
(load "../utilities.lisp")
(load "globals.lisp")
(load "tracking_vectors.lisp")
(load "action.lisp")
(load "reset_spacetime.lisp")
(load "generalized-hash-table-counting-functions.lisp")
(load "simplex.lisp")
(load "topological_checks.lisp")
(load "moves.lisp")
(load "ascii_plotting_tools.lisp")
(load "initialization.lisp")
(load "montecarlo.lisp")
(load "output.lisp")
\end{lstlisting}

\section{Getting a spacetime}

When you run a simulation, you're going to want to do one of two
tings: create a new spacetime to thermalize, or load an existing
spacetime from a file.

\subsection{Creating a new spacetime}

You can initialize your spacetime with the
``set-t-slcies-with-v-volume'' command. For example:

\begin{lstlisting}
(set-t-slices-with-v-volume :num-time-slices 64
                            :target-volume 30850
                            :spatial-topology "S2"
                            :boundary-conditions "OPEN"
               :initial-spatial-geometry "boundary_files/tetrahedron.boundary"
               :final-spatial-geometry "boundary_files/tetrahedron.boundary")
\end{lstlisting}

Some options require explanation. 
\begin{itemize}
\item The input number of time slices \textit{must} be even, due to
  the way the simulation initializes the spacetime. In the periodic
  boundary conditions case, the actual number of time slices the
  computer will generate is indeed the number you put here. However,
  in the fixed boundary conditions case, one additional time slice
  will be added.
\item The options for spatial topology are ``S2'' and ``T2,'' for the
  sphere and the torus respectively. The sphere topology works for
  everything. However, the torus only works for periodic boundary
  conditions.
\item Boundary condition types are ``OPEN'' and ``PERIODIC''.
\item The ``initial-spatial-geoemtry'' and ``final-spatial-geometry''
  fields are entirely optional. If you don't want them, don't even
  type them. Like so:
\begin{lstlisting}
(set-t-slices-with-v-volume :num-time-slices 64
			    :target-volume(* 8 1024)
			    :spatial-topology "S2"
			    :boundary-conditions "PERIODIC")
\end{lstlisting}
If you don't include these options and you use ``OPEN'' boundary
conditions, then the boundaries will be minimal triangulations of the
sphere: tetrahedra.
\item The input to the ``initial-spatial-geometry'' and
  ``final-spatial-geometry'' options should be a list of
  space-separated lists, each containing three points (for a triangle,
  you see). Using this format, we can completely describe a 2D
  triangulation. For instance, a tetrahedron is
\begin{lstlisting}
((4 3 2) (4 1 3) (1 4 2) (2 1 3))
\end{lstlisting}
Give each point a number, and draw the triangles, and you'll see this
is a tetrahedron. Generalize this format to input any boundary
geometry homeomorphic to the sphere.n
\end{itemize}

Once you've initialized a spacetime, you also need to set the coupling
constants. The two commands you can use for this are
``set-k0-k3-alpha'' and ``set-k-litl-alpha.'' They both set all
coupling constants, but in one case you directly set $k$ and $\lambda$
(called litL here) and in the other case you set $k_0$ and $k_3$, as
discussed in Amb\o rn and Loll. In theory, in the fixed boundary
conditions case, $k_0$ and $k_3$ should be radically different
functions of $k$ and $\lambda$ than they are in the periodic boundary
conditions case. However, so that we can easily compare $k$ and
$\lambda$ between simulations, the formulae for $k_0$ and $k_3$ are
identical to what they are in the fixed boundary conditions case.
$$\texttt{k} = k=\frac{1}{8\pi G}$$
and
$$\texttt{litL} = \lambda = k\Lambda.$$
$k_0=$k0 and $k_3$=k3 are related to $k$ and $\lambda$ in a
complicated way. $\alpha=$alpha is the length-squared of time-like
links. There is no benefit to making it anything but -1.

An example of the command syntax is 
\begin{lstlisting}
(set-k0-k3-alpha 1.0 0.75772 -1)
\end{lstlisting}
or
\begin{lstlisting}
(set-k-litl-alpha 1 5.0 -1)
\end{lstlisting}
The values are in order. The first input is k. The second is litL. The
third is alpha.

Once you've initialized the spacetime and set the coupling constants,
you're ready to run simulations.

\subsection{Loading an existing spacetime}
To load a spacetime from a file, run:
\begin{lstlisting}
(with-open-file (f "/path/to/filename.3sx2p1")
                (load-spacetime-from-file f))
\end{lstlisting}
The program will automatically build the spacetime defined in the file
and set the coupling constants to the correct values.

\section{Running a simulation}

Once you've loaded or generated a spacetime, you might want to change
a couple of parameters, by calling some commands like:
\begin{lstlisting}
(setf *eps* 0.05)
(setf SAVE-EVERY-N-SWEEPS 500)
(setf NUM-SWEEPS 50000)
\end{lstlisting}

These all work the same as in the previous versions of the CDT
code. The function ``setf'' changes a parameter as per common lisp
standards. 
\begin{itemize}
\item *eps*---The damping parameter. Changes the size of the critical surface.
\item SAVE-EVERY-N-SWEEPS---How often you save a spacetime if you're
  generateing multiple in a single run.
\item NUM-SWEEPS---The total number of sweeps the simulation goes
  through before finishing.
\end{itemize}

We also have a family of generate-* functions. These are the top-level
functions that perform (sweep). Each collects data for a particular
use-case. They are:
\begin{itemize}
\item \textbf{generate-data-console:} Periodically prints simplex
  counts and move acceptance ratios to the console. Good for tuning k0
  and k3 parameters. Outdated with the addition of
  ``map\_phase\_space\_parallelized.py''.
\item \textbf{generate-data:} Periodically re-writes a single file
  using save-spacetime-to-file and a second file with the current
  progress.
\item \textbf{generate-data-v2:} Like generate-data, but writes a new
  file each time so progress file is needed.
\item \textbf{generate-data-v3:} Like generate-data-v2, but generates
  spatial 2-simplex and 3-simplex information every
  SAVE-EVERY-N-SWEEPS.
\item \textbf{generate-movie-data:} Periodically appends to a file a
  new list of count-simplices-in-sandwich for every sandwich. Useful
  for visualizing the spacetime change over sweeps. Writes out a
  second file with progress data.
\item \textbf{generate-spacetime-and-movie-data:} This is a
  combination of generate-data and generate-movie-data.
\end{itemize}

\section{A complete script}
This is an example of a complete script you could run to generate a
new spacetime.
\begin{lstlisting}
(load "cdt2p1.lisp")
(setf NUM-SWEEPS 50000)
(initialize-t-slices-with-v-volume :num-time-slices          28
				   :target-volume            30850
				   :spatial-topology         "s2"
				   :boundary-conditions      "open"
        :initial-spatial-geometry "boundary_files/octahedron.boundary"
        :final-spatial-geometry   "boundary_files/octahedron.boundary")
(set-k0-k3-alpha 1.0 0.75772 -1)
(generate-spacetime-and-movie-data)
\end{lstlisting}

This is an example of a complete script you could run to load a
spacetime from a file and generate an ensemble of 1000
spacetimes. (Note that there's a better way to do this if you have
access to multiple cores. I'll get to that in a minute.)
\begin{lstlisting}
(load "cdt2p1.lisp")
(setf NUM-SWEEPS 500000)
(setf SAVE-EVERY-N-SWEEPS 500)
(with-open-file (f "/path/to/file.3sx2p1")
                (load-spacetime-from-file f))
(generate-data-v2)
\end{lstlisting}


\section{The fastest way to build an ensemble}
The user would be wise to generate a new spacetime and thermalize it
before generating an ensemble. We only want to add probable spacetimes
to our path integral. Once a spacetime has been thermalized, however,
we can take advantage of parallelization to speed up the process of
building an ensemble. The algorithm goes something like this:
\begin{enumerate}
\item Generate a spacetime from scratch and let it thermalize for a
  long time (maybe 50000 sweeps or so).
\item Run a simulation that loads that spacetime, uses
  generate-data-v2 and saves every 500 sweeps or so. Run said
  simulation as many times as you have CPU cores on your computer.
\end{enumerate}
I have written scripts to make this easier. The post-thermalization script is slihtly different than the load-spacetime-from-file listed above:
\begin{lstlisting}
;;;; default_post_thermalization.script.lisp
;;;; Author: Jonah Miller (jonah.maxwell.miller@gmail.com)

;;;; Loads a thermalized file and calculates how many sweeps are
;;;; necessary given parallelization.

(load "cdt2p1")

;;;; Constants
(defvar *filename* "/path/to/file.3sx2p1"
  "The file we will use.")
(defvar *num-cores* 28 "The number of cores we can run the simulation on.")

(setf SAVE-EVERY-N-SWEEPS 500) ; How different each element of the
			       ; ensemble should be.
(defvar *ensemble-size* 1000 "How many spacetimes we want total.")

(defvar *total-sweeps* (* *ensemble-size* SAVE-EVERY-N-SWEEPS) 
  "How many sweeps we need to get the ensemble we want.")
(defvar *sweeps-per-core* (ceiling (/ *total-sweeps* *num-cores*))
  "The number of sweeps used per core.")

(setf NUM-SWEEPS *sweeps-per-core*)

(with-open-file (f *filename*) (load-spacetime-from-file f))

(generate-data-v2)
\end{lstlisting}

You can find this exact file in
~/cdt\_scripts/default\_post\_thermalization.script.lisp. Obviously, it
needs to be changed depending on the number of cores the user has, the
ensemble size required, etc.. Don't bother running the script
*num-cores* times, though. Instead, use the shell script ~/start\_cores.sh.

\lstset{language=Bash,numbers=left,stepnumber=1,frame=shadowbox,rulesepcolor=\color{blue}}
\begin{lstlisting}
SCRIPTNAME="default_post_thermalization.script.lisp"
LOGFILENAME="ensemble_building.log"

for i in {1..28}; do 
    nice sbcl --dynamic-space-size 2000 --script $SCRIPTNAME >> $LOGFILENAME &
    sleep 2s
    echo $i
done

echo "All finished"

\end{lstlisting}
This script starts the script you put in SCRIPTNAME 28 times. Change
the number 28 to the number of cores you have. Then just run
./start\_cores.sh.

\section{Additional Programs}

There's a lot more to the CDT package than the simulations. There are
a number of data analysis tools under data\_analysis\_scripts. Those
files should be well documented in the comments---well enough to run
them. Also check out the other files in the documentation folder.

\end{document}